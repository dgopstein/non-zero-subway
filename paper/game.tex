\documentclass{acm_proc_article-sp}

\usepackage{graphicx}
\graphicspath{ {images/}, {} }

\begin{document}

\title{Subway Spot Selection}
\subtitle{A Game}

\numberofauthors{1} 
\author{
\alignauthor
       Dan Gopstein\\
       \affaddr{New York University}\\
       \email{dgopstein@nyu.edu}
}

\date{December 2014}

\maketitle
\begin{abstract}
We introduce the Subway Spot Selection Game, or Subway Game for short, a new continuous kernel sequential location game that gets its name for its similarity to subway seat selection.
\end{abstract}

\section{Introduction}
There is a rich history of studying location games as an analogy to economically relevant topics. The Hotelling game\cite{hotelling1990stability}, generally considered the first location game, investigated a duopoly controlled by the price model and one dimensional placement of two firms on a line. More recently this same model has been extended to multiple players\cite{economides1993hotelling}, to non-uniform metric spaces\cite{zhao2008isolation} and even the men's restroom\cite{heufer2011washroom, kranakis2010urinal}

\bibliographystyle{abbrv}
\bibliography{game}

%\balancecolumns 

\end{document}
