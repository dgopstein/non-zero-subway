\documentclass{acm_proc_article-sp}

\usepackage[usenames, dvipsnames]{color}
\usepackage{tikz}

\definecolor{lightgray}{gray}{0.9}


\newcommand{\startline} {
  \begin{tikzpicture}[scale = 8]
  \draw [gray] (0,0) -- (1,0);
}

\newcommand{\drawpoint}[2][black] {
  \filldraw [#1] (#2,0) circle (.5pt);
}

\newcommand{\stopline} {
  \end{tikzpicture}
}




\begin{document}

\title{Subway Spot Selection}
\subtitle{A Game}

\numberofauthors{1} 
\author{
\alignauthor
       Dan Gopstein\\
       \affaddr{New York University}\\
       \email{dgopstein@nyu.edu}
}

\date{December 2014}

\maketitle
\begin{abstract}
We introduce the Subway Spot Selection Game, or Subway Game for short, a new continuous kernel sequential location game that models the process of selecting a seat on a subway car as it fills up.
\end{abstract}

\section{Introduction}
There is a rich history of studying location games as an analogy to economically relevant topics. The Hotelling game\cite{hotelling1990stability}, generally considered the first location game, investigated a duopoly controlled by the price model and one dimensional placement of two firms on a line. More recently this same model has been extended to multiple players\cite{economides1993hotelling}, to non-uniform metric spaces\cite{zhao2008isolation} and even the men's restroom\cite{heufer2011washroom, kranakis2010urinal}.
\indent We now shift our attention to another important, yet distinctly different, domain. In the field of Urban Planning, public spaces are created by designers informed by common sense and experimental results, rarely are problems like subway passenger behavior investigated on a theoretical level. We would like to introduce our treatment of this Subway Game to help address the lack of theoretical information available to the designers of public transportation. As an analogy for passengers arranging themselves in a subway car we propose an isolation game on the unit line.

\section{The Subway Game}
\begin{itemize}
\item Let $S$ be a real-valued, closed, bounded set between [0, 1]
\item Let there be an unknown, unbounded number of players $N$
\item Let there be one stage $t$ for each players $P_i$
\item On stage $i$, player $P_i$ chooses a point $a_i$ in $S$
\item After each stage, each player with $i \leq t$ receives payoff $u_i(a_i)$
\item Let $u_i(a_i)=min(|a_i - a_j|)$ for all $j <  t$ and $j \neq i$
\item Let $u_1(a_1)=1.0$
\end{itemize}

\subsection{Example Play}
The first player in the Subway Game is presented with a blank unit line. Since no other players have chosen any points, $P_i$ is free to play where ever they would like and still receive payoff of $1.0$. Let's imagine $P_i$ chooses $a_1=3/4$

\startline \drawpoint[blue]{3/4} \stopline

$P_1$ receives payoff 1.0 because $u_1(1/3)=u_1(a_1)=1.0$. After the first stage is over, $t = 2$ and player $P_2$ is free to choose any point in S. Let's say $P_2$ chooses to play $1/3$.

\startline \drawpoint[blue]{1/3} \drawpoint{3/4} \stopline

After stage $t = 2$ has finished, $P_1$ and $P_2$ both receive a payoff of $u_1(a_1) = u_2(a_2) = |3/4 - 1/3| = 5/12$. Now assume one last player plays, and they choose $a_3 = 0$

\startline \drawpoint[blue]{0} \drawpoint{1/3} \drawpoint{3/4} \stopline

After stage $t = 3$ the payoffs go as follows: $u_2(a_2) = u_3(a_3) = |0 - 1/3| = 1/3$ and $P_1$ receives the same payoff as in $t = 2$ $u_1(a_1) = |3/4 - 1/3| = 5/12$

In total each player has received the following utility:
\begin{description}
\item[$P_1$:] 1 + 5/12 + 5/12 = 11/6
\item[$P_2$:] 5/12 + 1/3 = 3/4
\item[$P_3$:] 1/3 = 1/3
\end{description}

From this example it is clear that the first two players have quite a bit of regret. For example, assuming $P_2$ and $P_3$ would both make the same choices, but $P_1$ instead chose 1, their second and third payoffs would've both been 2/3 instead of 5/12, and their total utility gain would've been 7/3 instead of 11/6.

\startline \drawpoint{0} \drawpoint{1/3} \drawpoint[lightgray]{3/4} \drawpoint[Green]{1} \stopline

In the following sections we consider an example more geared towards optimality, and then we will investigate a general optimal strategy under certain assumptions.

\subsection{An optimal example}

\bibliographystyle{abbrv}
\bibliography{game}

%\balancecolumns 

\end{document}
