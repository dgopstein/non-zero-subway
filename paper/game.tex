\documentclass{acm_proc_article-sp}

\usepackage[usenames, dvipsnames]{color}
\usepackage{tikz}

\definecolor{lightgray}{gray}{0.9}


\newcommand{\startline} {
  \begin{tikzpicture}[scale = 8]
  \draw [gray] (0,0) -- (1,0);
}

\newcommand{\drawpoint}[2][black] {
  \filldraw [#1] (#2,0) circle (.5pt);
}

\newcommand{\stopline} {
  \end{tikzpicture}
}




\begin{document}

\title{Subway Spot Selection}
\subtitle{A Game}

\numberofauthors{1} 
\author{
\alignauthor
       Dan Gopstein\\
       \affaddr{New York University}\\
       \email{dgopstein@nyu.edu}
}

\date{December 2014}

\maketitle
\begin{abstract}
We introduce the Subway Spot Selection Game, or Subway Game for short, a new continuous kernel sequential location game that models the process of selecting a seat on a subway car as it fills up.
\end{abstract}

\section{Introduction}
There is a rich history of studying location games as an analogy to economically relevant topics. The Hotelling game\cite{hotelling1990stability}, generally considered the first location game, investigated a duopoly controlled by the price model and one dimensional placement of two firms on a line. More recently this same model has been extended to multiple players\cite{economides1993hotelling}, to non-uniform metric spaces\cite{zhao2008isolation} and even the men's restroom\cite{heufer2011washroom, kranakis2010urinal}.
\indent We now shift our attention to another important, yet distinctly different, domain. In the field of Urban Planning, public spaces are created by designers informed by common sense and experimental results, rarely are problems like subway passenger behavior investigated on a theoretical level. We would like to introduce our treatment of this Subway Game to help address the lack of theoretical information available to the designers of public transportation. As an analogy for passengers arranging themselves in a subway car we propose an isolation game on the unit line.

\section{The Subway Game}
\begin{itemize}
\item Let $S$ be a real-valued, closed, bounded set between [0, 1]
\item Let there be an unknown, unbounded number of players $N$
\item Let there be one stage $t$ for each players $P_i$
\item On stage $i$, player $P_i$ chooses a point $a_i$ in $S$
\item After each stage, each player with $i \leq t$ receives payoff $u_i(a_i)$
\item Let $u_i(a_i)=min(|a_i - a_j|)$ for all $j <  t$ and $j \neq i$
\item Let $u_1(a_1)=1$
\end{itemize}

\subsection{Example Play}
The first player in the Subway Game is presented with a blank unit line. Since no other players have chosen any points, $P_i$ is free to play where ever they would like and still receive payoff of $1$. Let's imagine $P_i$ chooses $a_1=3/4$

\startline \drawpoint[blue]{3/4} \stopline

$P_1$ receives payoff 1 because $u_1(1/3)=u_1(a_1)=1$. After the first stage is over, $t = 2$ and player $P_2$ is free to choose any point in S. Let's say $P_2$ chooses to play $1/3$.

\startline \drawpoint[blue]{1/3} \drawpoint{3/4} \stopline

After stage $t = 2$ has finished, $P_1$ and $P_2$ both receive a payoff of $u_1(a_1) = u_2(a_2) = |3/4 - 1/3| = 5/12$. Now assume one last player plays, and they choose $a_3 = 0$

\startline \drawpoint[blue]{0} \drawpoint{1/3} \drawpoint{3/4} \stopline

After stage $t = 3$ the payoffs go as follows: $u_2(a_2) = u_3(a_3) = |0 - 1/3| = 1/3$ and $P_1$ receives the same payoff as in $t = 2$ $u_1(a_1) = |3/4 - 1/3| = 5/12$

In total each player has received the following utility:
\begin{description}
\item[$P_1$:] 1 + 5/12 + 5/12 = 11/6
\item[$P_2$:] 5/12 + 1/3 = 3/4
\item[$P_3$:] 1/3 = 1/3
\end{description}

From this example it is clear that the first two players have quite a bit of regret. For example, assuming $P_2$ and $P_3$ would both make the same choices, but $P_1$ instead chose 1, their second and third payoffs would've both been 2/3 instead of 5/12, and their total utility gain would've been 7/3 instead of 11/6.

\startline \drawpoint{0} \drawpoint{1/3} \drawpoint[lightgray]{3/4} \drawpoint[Green]{1} \stopline

In the following sections we consider an example more geared towards optimality, and then we will investigate a general optimal strategy under certain assumptions.

\subsection{An Optimal Example}

Now we will look at a second game, one where each of the players has learned from mistakes in the first game. As we saw at the end of the last section, in order to minimize historical regret $P_i$ should choose $a_i=1$, in fact, for reasons we will show later, this is always an optimal action for the first player.

\startline \drawpoint[blue]{1} \stopline

Next, $P_2$, feeling more compelled to respond well, decides that they can do better than their choice of 1/3 in the first game. $P_2$ recognizes that in terms of the $t = 2$ payoff and choice $a_2 > 1/3$ would decrease $u_2$ and any $a_2 < 1/3$ would increase it. On the other hand, a choice of $a_2 < 1/3$ would decrease the distance (and therefore payoff) for $P_2$ in stage $t = 3$ if $P_3$ were to make the decision to play 0 again. However, $P_2$ intuits that if they were to play $a_2 = 0$ they could get the best of both worlds, they would receive $u_2(0) = 1$ this turn, and while its not obvious where $P_3$ would play, its clear that $P_3$ would have no incentive to play the same point as $P_2$. So $P_2$ plays $a_2 = 0$ and hopes for the best in $t = 3$. It turns out that this too is an optimal play for $P_2$ for a similar reason that $a_1 = 1$ was.

\startline \drawpoint[blue]{0} \drawpoint{1} \stopline

As $P_3$ gets ready to make their move, the realize, as $P_2$ predicted, that their old choice of $a_3 = 0$ is now the least desirable choice possible. Since the last game ended in $t = 3$ and $P_3$ suspects that they might get lucky and this example might just end after 3 stages as well, they decide to not put too much effort into their decision and chose the one-stage distance maximum by moving their knife\cite{dubins1961cut} over all the points in S until they find the maximum at 1/2, which they choose.

\startline \drawpoint{0} \drawpoint[blue]{1/2} \drawpoint{1} \stopline

As it turns out, $P_3$ was right about several things. Firstly, $a_3$, in addition to offering the maximal one-stage payoff at $t = 3$ also serves as the long-term optimal strategy for that stage. $P_3$ was also correct that there will be no more stages in this example, but before we go on to prove the optimality of each of the claims we made above, let's quickly tally up the scores of the above example:
\begin{description}
\item[$P_1$:] 1 + 1 + 1/2 = 5/2
\item[$P_2$:] 1 + 1/2 = 3/2
\item[$P_3$:] 1/2 = 1/2
\end{description}

In closing, it is interesting to note that in addition to being an individual benefit to each player over the previous example, it is both a Nash equilibrium and a Pareto optimal configuration.


\bibliographystyle{abbrv}
\bibliography{game}

%\balancecolumns 

\end{document}
